\documentclass[12pt,letterpaper]{book}


\title{\textbf{HOURIN's KEQWM MANUAL}}
\author{Hourin} 

\begin{document}
\maketitle

\section{Introduction}
My online alias is Hourin, Satori Hourin. I love dwm for its simplicity
and its unique way to configure and modify things (editing
the source code then recompile). It enables a lot more control and freedom
for the user, expanding beyond the cringe config menus/files that
most Desktop Environments/Window Managers had. \\
The reason for this is because dwm is supposed to be "suckless". It doesn't need
a bloated graphical GUI for customizing that can mimic a fraction of the
real man's way to configure stuff, neither does it need a new or to use troublesome
syntaxes for a config file. No. All it takes to configure dwm is knowledge of the C
programming language, which, let's be honest, virtually all Linux users know. \\
So anyway, dwm is a dynamic window manager for Xorg. Dynamic, because it's both
a tiling window manager and a floating/stacking window manager. dwm's customizability is
undeniable. Customizing it by editing and compiling the source code gives you 100\% control
over everything. You can change any colors, any text strings, any functions, adding new
layouts, everything. Also, by using dwm, you would earn the status of a more invested Linux
user, giving you more privilages in r/unixporn or Linux-related places. \\
\\
keqwm is my very own customized dwm. It has few more key shortcuts, some changes to the
color, how the bar is drawn (without using a bloated bar patch). I also added a few important
patches to it, for example, fibonacci and centeredmaster (though I'm planning on adding
swallow, tatami, and alpha patch soon). This manual guides you how to use it.

\section{Key bindings}
Note: in this section, I use the Emacs-like key notation:
C-key: pressing key while holding Ctrl.
M-key: pressing key while holding Meta/Alt.
S-key: pressing key while holding Super/Window.
If key is capitalized then you know you have to press Shift.


\subsection{Programs}
M-RETURN: terminal (Alacritty by default) \\
S-b: browser (Firefox by default) \\
S-s: terminal with htop (alacritty -e htop) \\
S-v: terminal with alsamixer \\
S-return: dmenu \\
S-e: terminal with Vim \\
S-E: terminal with Emacs \\
S-d: terminal with vifm \\
S-D: file manager (thunar) \\


\subsection{Layouts} //FIXME: pdflatex has behavious i didn't know existed
M-t: tile (default)		[]= \\
M-f: float (default)		\>\<\> \\
M-m: monocle (default)		[M] \\
M-u: centered master		|M| \\
M-U: centered floating master	\<M\< \\
S-f: fibonnaci - dwindle	[/\] \\
S-F: fibonnaci - spiral		(@) \\


\subsection{Controls}
S-RETURN: float/unfloat a window \\
S-x: kill selected window \\
M-Q: kill keqwm \\
S-[arrow keys]: move selected window if floating \\
S-[ARROW KEYS]: resize selected window if floating \\
M-j: switch window \\
M-k: switch window \\
M-h: decrease master stack size \\
M-l: increase master stack size \\
M-i: increase number of masters \\
M-d: decrease number of masters \\
M-[number]: go to tag number \\
M-[NUMBER]: move window to tag number \\
S-1: mute volume \\
S-2: volume down \\
S-3: volume up \\
M-b: toggle bar \\























\end{document}
